\chapter*{Preface}
\addcontentsline{toc}{chapter}{Preface}  

\section*{README}

These notes are a complement to the Lecture Notes for STAT3001 
Semester 1 2022 and are largely quoted or paraphrased from 
the references notes written by Dirk Kroese and adapted by Geoff McLachlan 
for the Semester 2 2020 running of STAT2004, plus other sources reference throughout. 
You can access these notes on blackboard by searching for the 2020 running of this course. 
\bigskip
The lecture notes assume knowledge of the basics of probability, 
however, statistical inference hinges upon many basic 
probability notions. The second half of these notes contain the foundations 
of probability and go through some more rigorous formulas 
and theorems that will be needed. 
Well, I did some of that stuff since I think sigma algebra notation is useful, however, 
I did end up abandoning making that section of notes 
since Dirk/Geoff's original reference notes are quite comprehensive anyway. 
The only issue with referring to the 2020 Reference notes is that Geoff uses vectors for every formula in 2021, so that's something to keep in mind when referring to the 2020 notes. 
\bigskip
These notes also provide pointers to where concepts have been used in tutorial or assignment questions. 
It's not rigorous right now bc I haven't been very diligent with citing that stuff but I aim for it to be rigorous. 
\bigskip
I'd like to note that these notes are wonderful but they 
took a lot of time. 
If you know me, shout me a coffee sometime because I don't have a ko-fi hahaha. 

\section*{How to read my notes}

Here are some of the different text environments you will encounter:
\begin{theorem}[Theorem name]
    The statement of a theorem. 
\end{theorem}

\begin{proof}%[Theorem name or definition]
    Here is where we might prove a property or theorem. 
\end{proof}


\begin{definition}[Term being defined]
    Here is a definition of a term. 
\end{definition}
Many terms will be introduced in the context of discussion, 
in which the key term will be \textbf{bolded} like so. 
This is done especially for simpler ideas which may have 
fewer mathematical properties to be investigated 
or used. 

\begin{example}[Name of example]
    Examples will either be presented as individual 
    examples or as a list of examples. 
\end{example}
Similarly, non-rigorous examples may simply be enumerated 
rather than given their own example environment. 
\begin{enumerate}
    \item Here is a simple example. 
    \item And here is another. 
    \item Since these are simple, it would not be 
    of much benefit to give them a reference of their own.
\end{enumerate}
\begin{exercise}[Name of exercise]
    These are exercises left to the reader. I may add solutions 
    in the appendix but ceebs tbh. These are also environments I may used to flag where a concept was used in a tutorial or assignment. 
\end{exercise}

These are self-explanatory for the most part. 
Also, most things in this document are hyperlinked :)